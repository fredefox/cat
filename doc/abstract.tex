\chapter*{Abstract}
The usual notion of propositional equality in intensional type-theory
is restrictive. For instance it does not admit functional
extensionality or univalence. This poses a severe limitation on both
what is \emph{provable} and the \emph{re-usability} of proofs. Recent
developments have, however, resulted in cubical type theory which
permits a constructive proof of these two important notions. The
programming language Agda has been extended with capabilities for
working in such a cubical setting. This thesis will explore the
usefulness of this extension in the context of category theory.

The thesis will motivate and explain why propositional equality in
cubical Agda is more expressive than in standard Agda. Alternative
approaches to Cubical Agda will be presented and their pros and cons
will be explained. It will emphasize why it is useful to have a
constructive interpretation of univalence. As an example of this two
formulations of monads will be presented: Namely monads in the
monoidal form and monads in the Kleisli form.

Finally the thesis will explain the challenges that a developer will
face when working with cubical Agda and give some techniques to
overcome these difficulties. It will also try to suggest how further
work can help alleviate some of these challenges.
