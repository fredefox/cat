\documentclass[a4paper]{beamer}
%% \documentclass[a4paper,handout]{beamer}
%% \usecolortheme[named=seagull]{structure}

\usepackage[utf8]{inputenc}

\usepackage{natbib}
\usepackage[hidelinks]{hyperref}

\usepackage{graphicx}

\usepackage{parskip}
\usepackage{multicol}
\usepackage{amssymb,amsmath,amsthm,stmaryrd,mathrsfs,wasysym}
\usepackage[toc,page]{appendix}
\usepackage{xspace}

% \setlength{\parskip}{10pt}

% \usepackage{tikz}
% \usetikzlibrary{arrows, decorations.markings}

% \usepackage{chngcntr}
% \counterwithout{figure}{section}

\usepackage{listings}
\usepackage{fancyvrb}

\usepackage{chalmerstitle}

\usepackage{mathpazo}
\usepackage[scaled=0.95]{helvet}
\usepackage{courier}
\linespread{1.05} % Palatino looks better with this

\usepackage{fontspec}
\setmonofont[Mapping=tex-text]{FreeMono.otf}
%% \setmonofont{FreeMono.otf}


\pagestyle{fancyplain}
\setlength{\headheight}{15pt}
\renewcommand{\chaptermark}[1]{\markboth{\textsc{Chapter \thechapter. #1}}{}}
\renewcommand{\sectionmark}[1]{\markright{\textsc{\thesection\ #1}}}


\newcommand{\coloneqq}{\mathrel{\vcenter{\baselineskip0.5ex \lineskiplimit0pt
                     \hbox{\scriptsize.}\hbox{\scriptsize.}}}%
                     =}

\newcommand{\defeq}{\coloneqq}
\newcommand{\bN}{\mathbb{N}}
\newcommand{\bC}{\mathbb{C}}
\newcommand{\bX}{\mathbb{X}}
% \newcommand{\to}{\rightarrow}
\newcommand{\mto}{\mapsto}
\newcommand{\UU}{\ensuremath{\mathcal{U}}\xspace}
\let\type\UU
\newcommand{\nomen}[1]{\emph{#1}}
\newcommand{\todo}[1]{\textit{#1}}
\newcommand{\comp}{\circ}
\newcommand{\x}{\times}
\newcommand{\Hom}{\mathit{Hom}}
\newcommand{\fmap}{\mathit{fmap}}
\newcommand{\idFun}{\mathit{id}}
\newcommand{\Sets}{\mathit{Sets}}
\newcommand{\Set}{\mathit{Set}}
\newcommand{\MCU}{\UU}
\newcommand{\isSet}{\mathit{isSet}}
\newcommand{\tp}{\;\mathord{:}\;}
\newcommand{\subsubsubsection}[1]{\textbf{#1}}
\newcommand{\WIP}[1]{\textbf{[WIP]}}


\title{Univalent Categories}
\subtitle{A formalization of category theory in Cubical Agda}

\newcommand{\myname}{Frederik Hangh{\o}j Iversen}
\author[\myname]{
  \myname\\
  \footnotesize Supervisors: Thierry Coquand, Andrea Vezzosi\\
  Examiner: Andreas Abel
}
\institute{Chalmers University of Technology}

\begin{document}

\frame{\titlepage}

\begin{frame}
\frametitle{Introduction}
Category Theory: The study of abstract functions. Slogan: ``It's the
arrows that matter''\pause

Objects are equal ``up to isomorphism''.  Univalence makes this notion
precise.\pause

Agda does not permit proofs of univalence.  Cubical Agda admits
this.\pause

Goal: Construct a category whose terminal objects are (equivalent to)
products. Use this to conclude that products are propositions, not a
structure on a category.
\end{frame}

\begin{frame}
\frametitle{Outline}
The path type

Definition of a (pre-) category

1-categories

Univalent (proper) categories

The category of spans
\end{frame}

\section{Paths}
\begin{frame}
  \frametitle{Paths}
  \framesubtitle{Definition}
  Heterogeneous paths
  \begin{equation*}
    \Path \tp (P \tp \I → \MCU) → P\ 0 → P\ 1 → \MCU
  \end{equation*}
  \pause
  For $P \tp \I → \MCU$ and $a_0 \tp P\ 0$, $a_1 \tp P\ 1$
  inhabitants of $\Path\ P\ a_0\ a_1$ are like functions
  %
  $$
  p \tp ∏_{i \tp \I} P\ i
  $$
  %
  Which satisfy $p\ 0 & = a_0$ and $p\ 1 & = a_1$
  \pause

  Homogenous paths
  $$
  a_0 ≡ a_1 ≜ \Path\ (\var{const}\ A)\ a_0\ a_1
  $$
\end{frame}

\begin{frame}
  \frametitle{Pre categories}
  \framesubtitle{Definition}
  Data:
  \begin{align*}
    \Object   & \tp \Type \\
    \Arrow    & \tp \Object → \Object → \Type \\
    \identity & \tp \Arrow\ A\ A \\
    \llll      & \tp \Arrow\ B\ C → \Arrow\ A\ B → \Arrow\ A\ C
  \end{align*}
  %
  \pause
  Laws:
  %
  \begin{align*}
  \var{isAssociative} & \tp
  h \llll (g \llll f) ≡ (h \llll g) \llll f \\
  \var{isIdentity} & \tp
  (\identity \llll f ≡ f)
  ×
  (f \llll \identity ≡ f)
  \end{align*}

\end{frame}
\begin{frame}
  \frametitle{Pre categories}
  \framesubtitle{1-categories}
Cubical Agda does not admit \emph{Uniqueness of Identity Proofs}
(UIP).  Rather there is a hierarchy of \emph{Homotopy Types}:
Contractible types, mere propositions, sets, groupoids, \dots

\pause
  1-categories:
  $$
  \isSet\ (\Arrow\ A\ B)
  $$
\pause
  \begin{align*}
    \isSet    & \tp \MCU → \MCU \\
    \isSet\ A & ≜ ∏_{a_0, a_1 \tp A} \isProp\ (a_0 ≡ a_1)
  \end{align*}
\end{frame}

\begin{frame}
\frametitle{Outline}
The path type \ensuremath{\checkmark}

Definition of a (pre-) category \ensuremath{\checkmark}

1-categories \ensuremath{\checkmark}

Univalent (proper) categories

The category of spans
\end{frame}

\begin{frame}
  \frametitle{Categories}
  \framesubtitle{Univalence}
  Let $\approxeq$ denote isomorphism of objects.  We can then construct
  the identity isomorphism in any category:
  $$
  (\identity , \identity , \var{isIdentity}) \tp A \approxeq A
  $$
  \pause
  Likewise since paths are substitutive we can promote a path to an isomorphism:
  $$
  \idToIso \tp A ≡ B → A ≊ B
  $$
  \pause
  For a category to be univalent we require this to be an equivalence:
  %
  $$
  \isEquiv\ (A ≡ B)\ (A \approxeq B)\ \idToIso
  $$
  %
\end{frame}
\begin{frame}
  \frametitle{Categories}
  \framesubtitle{Univalence, cont'd}
  $$\isEquiv\ (A ≡ B)\ (A \approxeq B)\ \idToIso$$
  \pause%
  $$(A ≡ B) ≃ (A \approxeq B)$$
  \pause%
  $$(A ≡ B) ≅ (A \approxeq B)$$
  \pause%
  Name the inverse of $\idToIso$:
  $$\isoToId \tp (A \approxeq B) → (A ≡ B)$$
\end{frame}
\begin{frame}
  \frametitle{Propositionality of products}
  Construct a category for which it is the case that the terminal
  objects are equivalent to products:
  \begin{align*}
    \var{Terminal} ≃ \var{Product}\ ℂ\ 𝒜\ ℬ
  \end{align*}

  \pause
  And since equivalences preserve homotopy levels we get:
  %
  $$
  \isProp\ \left(\var{Product}\ \bC\ 𝒜\ ℬ\right)
  $$
\end{frame}

\begin{frame}
  \frametitle{Categories}
  \framesubtitle{A theorem}
  %
  Let the isomorphism $(ι, \inv{ι}, \var{inv}) \tp A \approxeq B$.
  %
  \pause
  %
  The isomorphism induces the path
  %
  $$
  p ≜ \isoToId\ (\iota, \inv{\iota}, \var{inv}) \tp A ≡ B
  $$
  %
  \pause
  and consequently a path on arrows:
  %
  $$
  p_{\var{dom}} ≜ \congruence\ (λ x → \Arrow\ x\ X)\ p
  \tp
  \Arrow\ A\ X ≡ \Arrow\ B\ X
  $$
  %
  \pause
  The proposition is:
  %
  \begin{align}
    \label{eq:coeDom}
    \tag{$\var{coeDom}$}
    ∏_{f \tp A → X}
    \var{coe}\ p_{\var{dom}}\ f ≡ f \llll \inv{\iota}
  \end{align}
\end{frame}
\begin{frame}
  \frametitle{Categories}
  \framesubtitle{A theorem, proof}
  \begin{align*}
    \var{coe}\ p_{\var{dom}}\ f
    & ≡ f \llll (\idToIso\ p)_1 && \text{By path-induction} \\
    & ≡ f \llll \inv{\iota}
    && \text{$\idToIso$ and $\isoToId$ are inverses}\\
  \end{align*}
  \pause
  %
  Induction will be based at $A$.  Let $\widetilde{B}$ and $\widetilde{p}
  \tp A ≡ \widetilde{B}$ be given.
  %
  \pause
  %
  Define the family:
  %
  $$
  D\ \widetilde{B}\ \widetilde{p} ≜
  \var{coe}\ \widetilde{p}_{\var{dom}}\ f
  ≡
  f \llll \inv{(\idToIso\ \widetilde{p})}
  $$
  \pause
  %
  The base-case becomes:
  $$
  d \tp D\ A\ \refl =
  \left(\var{coe}\ \refl_{\var{dom}}\ f ≡ f \llll \inv{(\idToIso\ \refl)}\right)
  $$
\end{frame}
\begin{frame}
  \frametitle{Categories}
  \framesubtitle{A theorem, proof, cont'd}
  $$
  d \tp
  \var{coe}\ \refl_{\var{dom}}\ f ≡ f \llll \inv{(\idToIso\ \refl)}
  $$
  \pause
  \begin{align*}
    \var{coe}\ \refl_{\var{dom}}\ f
    & =
    \var{coe}\ \refl\ f \\
    & ≡ f
    && \text{neutral element for $\var{coe}$}\\
    & ≡ f \llll \identity \\
    & ≡ f \llll \var{subst}\ \refl\ \identity
    && \text{neutral element for $\var{subst}$}\\
    & ≡ f \llll \inv{(\idToIso\ \refl)}
    && \text{By definition of $\idToIso$}\\
  \end{align*}
  \pause
  In conclusion, the theorem is inhabited by:
  $$
  \var{pathInd}\ D\ d\ B\ p
  $$
\end{frame}
\begin{frame}
  \frametitle{Span category} \framesubtitle{Definition} Given a base
  category $\bC$ and two objects in this category $\pairA$ and $\pairB$
  we can construct the \nomenindex{span category}:
  %
  \pause
  Objects:
  $$
  ∑_{X \tp Object} (\Arrow\ X\ \pairA) × (\Arrow\ X\ \pairB)
  $$
  \pause
  %
  Arrows between objects $(A , a_{\pairA} , a_{\pairB})$ and
  $(B , b_{\pairA} , b_{\pairB})$:
  %
  $$
  ∑_{f \tp \Arrow\ A\ B}
  (b_{\pairA} \llll f ≡ a_{\pairA}) ×
  (b_{\pairB} \llll f ≡ a_{\pairB})
  $$
\end{frame}
\begin{frame}
  \frametitle{Span category}
  \framesubtitle{Univalence}
  \begin{align*}
    (X , x_{𝒜} , x_{ℬ}) ≡ (Y , y_{𝒜} , y_{ℬ})
  \end{align*}
  \begin{align*}
    \begin{split}
      p \tp & X ≡ Y \\
      & \Path\ (λ i → \Arrow\ (p\ i)\ 𝒜)\ x_{𝒜}\ y_{𝒜} \\
      & \Path\ (λ i → \Arrow\ (p\ i)\ ℬ)\ x_{ℬ}\ y_{ℬ}
    \end{split}
  \end{align*}
  \begin{align*}
    \begin{split}
      \var{iso} \tp & X \approxeq Y \\
      & \Path\ (λ i → \Arrow\ (\widetilde{p}\ i)\ 𝒜)\ x_{𝒜}\ y_{𝒜} \\
      & \Path\ (λ i → \Arrow\ (\widetilde{p}\ i)\ ℬ)\ x_{ℬ}\ y_{ℬ}
    \end{split}
  \end{align*}
  \begin{align*}
    (X , x_{𝒜} , x_{ℬ}) ≊ (Y , y_{𝒜} , y_{ℬ})
  \end{align*}
\end{frame}
\begin{frame}
  \frametitle{Span category}
  \framesubtitle{Univalence, proof}
  %
  \begin{align*}
    %% (f, \inv{f}, \var{inv}_f, \var{inv}_{\inv{f}})
    %% \tp
    (X, x_{𝒜}, x_{ℬ}) \approxeq (Y, y_{𝒜}, y_{ℬ})
    \to
    \begin{split}
      \var{iso} \tp & X \approxeq Y \\
      & \Path\ (λ i → \Arrow\ (\widetilde{p}\ i)\ 𝒜)\ x_{𝒜}\ y_{𝒜} \\
      & \Path\ (λ i → \Arrow\ (\widetilde{p}\ i)\ ℬ)\ x_{ℬ}\ y_{ℬ}
    \end{split}
  \end{align*}
  \pause
  %
  Let $(f, \inv{f}, \var{inv}_f, \var{inv}_{\inv{f}})$ be an inhabitant
  of the antecedent.\pause

  Projecting out the first component gives us the isomorphism
  %
  $$
  (\fst\ f, \fst\ \inv{f}
  , \congruence\ \fst\ \var{inv}_f
  , \congruence\ \fst\ \var{inv}_{\inv{f}}
  )
  \tp X \approxeq Y
  $$
  \pause
  %
  This gives rise to the following paths:
  %
  \begin{align*}
    \begin{split}
      \widetilde{p} & \tp X ≡ Y \\
      \widetilde{p}_{𝒜} & \tp \Arrow\ X\ 𝒜 ≡ \Arrow\ Y\ 𝒜 \\
    \end{split}
  \end{align*}
  %
\end{frame}
\begin{frame}
  \frametitle{Span category}
  \framesubtitle{Univalence, proof, cont'd}
  It remains to construct:
  %
  \begin{align*}
    \begin{split}
      & \Path\ (λ i → \widetilde{p}_{𝒜}\ i)\ x_{𝒜}\ y_{𝒜}
    \end{split}
  \end{align*}
  \pause
  %
  This is achieved with the following lemma:
  %
  \begin{align*}
    ∏_{q \tp A ≡ B} \var{coe}\ q\ x_{𝒜} ≡ y_{𝒜}
    →
    \Path\ (λ i → q\ i)\ x_{𝒜}\ y_{𝒜}
  \end{align*}
  %
  Which is used without proof.\pause

  So the construction reduces to:
  %
  \begin{align*}
    \var{coe}\ \widetilde{p}_{𝒜}\ x_{𝒜} ≡ y_{𝒜}
  \end{align*}%
  \pause%
  This is proven with:
  %
  \begin{align*}
    \var{coe}\ \widetilde{p}_{𝒜}\ x_{𝒜}
    & ≡ x_{𝒜} \llll \fst\ \inv{f} && \text{\ref{eq:coeDom}} \\
    & ≡ y_{𝒜} && \text{Property of span category}
  \end{align*}
\end{frame}
\begin{frame}
  \frametitle{Propositionality of products}
  We have
  %
  $$
  \isProp\ \var{Terminal}
  $$\pause
  %
  We can show:
  \begin{align*}
    \var{Terminal} ≃ \var{Product}\ ℂ\ 𝒜\ ℬ
  \end{align*}
  \pause
  And since equivalences preserve homotopy levels we get:
  %
  $$
  \isProp\ \left(\var{Product}\ \bC\ 𝒜\ ℬ\right)
  $$
\end{frame}
\begin{frame}
  \frametitle{Monads}
  \framesubtitle{Monoidal form}
  %
  \begin{align*}
    \EndoR  & \tp \Functor\ ℂ\ ℂ \\
    \pureNT
    & \tp \NT{\widehat{\identity}}{\EndoR} \\
    \joinNT
    & \tp \NT{(\EndoR \oplus \EndoR)}{\EndoR}
  \end{align*}
  \pause
  %
  Let $\fmap$ be the map on arrows of $\EndoR$.
  %
  \begin{align*}
    \join \llll \fmap\ \join
    & ≡ \join \llll \join \\
    \join \llll \pure\           & ≡ \identity \\
    \join \llll \fmap\     \pure & ≡ \identity
  \end{align*}
\end{frame}
\begin{frame}
  \frametitle{Monads}
  \framesubtitle{Kleisli form}
  %
  \begin{align*}
    \omapR & \tp \Object → \Object \\
    \pure  & \tp % ∏_{X \tp Object}
    \Arrow\ X\ (\omapR\ X) \\
    \bind  & \tp
    \Arrow\ X\ (\omapR\ Y)
    \to
    \Arrow\ (\omapR\ X)\ (\omapR\ Y)
  \end{align*}\pause
  %
  \begin{align*}
    \fish & \tp
    \Arrow\ A\ (\omapR\ B)
    →
    \Arrow\ B\ (\omapR\ C)
    →
    \Arrow\ A\ (\omapR\ C) \\
    f \fish g & ≜ f \rrrr (\bind\ g)
  \end{align*}
  \pause
  %
  \begin{align*}
    \bind\ \pure & ≡ \identity_{\omapR\ X} \\
    \pure \fish f & ≡ f \\
    (\bind\ f) \rrrr (\bind\ g) & ≡ \bind\ (f \fish g)
  \end{align*}
\end{frame}
\begin{frame}
  \frametitle{Monads}
  \framesubtitle{Equivalence}
  In the monoidal formulation we can define $\bind$:
  %
  $$
  \bind\ f ≜ \join \llll \fmap\ f
  $$
  \pause
  %
  And likewise in the Kleisli formulation we can define $\join$:
  %
  $$
  \join ≜ \bind\ \identity
  $$
  \pause
  The laws are logically equivalent. Since logical equivalence is
  enough for as an equivalence of types for propositions we get:
  %
  $$
  \var{Monoidal} ≃ \var{Kleisli}
  $$
  %
\end{frame}
\end{document}
