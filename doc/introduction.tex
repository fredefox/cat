\chapter{Introduction}
\section{Motivating examples}
%
In the following two sections I present two examples that illustrate some
limitations inherent in ITT and -- by extension -- Agda.
%
\subsection{Functional extensionality}
\label{sec:functional-extensionality}%
Consider the functions:
%
\begin{multicols}{2}
  \noindent
  \begin{equation*}
    f \defeq (n \tp \bN) \mto (0 + n \tp \bN)
  \end{equation*}
  \begin{equation*}
    g \defeq (n \tp \bN) \mto (n + 0 \tp \bN)
  \end{equation*}
\end{multicols}
%
$n + 0$ is \nomen{definitionally} equal to $n$, which we write as $n + 0 = n$.
This is also called \nomen{judgmental} equality. We call it definitional
equality because the \emph{equality} arises from the \emph{definition} of $+$
which is:
%
\begin{align*}
  +           & \tp \bN \to \bN \to \bN      \\
  n + 0       & \defeq n                   \\
  n + (\suc{m}) & \defeq \suc{(n + m)}
\end{align*}
%
Note that $0 + n$ is \emph{not} definitionally equal to $n$. The term $0 + n$ is in
normal form. I.e.; there is no rule for $+$ whose left-hand-side matches this
expression. We \emph{do}, however, have that they are \nomen{propositionally}
equal, which we write as $n + 0 \equiv n$. Propositional equality means that
there is a proof that exhibits this relation. Since equality is a transitive
relation we have that $n + 0 \equiv 0 + n$.

Unfortunately we don't have $f \equiv g$.\footnote{Actually showing this is
outside the scope of this text. Essentially it would involve giving a model
for our type theory that validates all our axioms but where $f \equiv g$ is
not true.} There is no way to construct a proof asserting the obvious
equivalence of $f$ and $g$ -- even though we can prove them equal for all
points. This is exactly the notion of equality of functions that we are
interested in; that they are equal for all inputs. We call this
\nomen{point-wise equality}, where the \emph{points} of a function refers
to its arguments.

In the context of category theory functional extensionality is e.g.\ needed to
show that representable functors are indeed functors. The representable functor
for a category $\bC$ and a fixed object in $A \in \bC$ is defined to be:
%
\begin{align*}
\fmap \defeq X \mto \Hom_{\bC}(A, X)
\end{align*}
%
The proof obligation that this satisfies the identity law of functors
($\fmap\ \idFun \equiv \idFun$) thus becomes:
%
\begin{align*}
\Hom(A, \idFun_{\bX}) = (g \mto \idFun \comp g) \equiv \idFun_{\Sets}
\end{align*}
%
One needs functional extensionality to ``go under'' the function arrow and apply
the (left) identity law of the underlying category to prove $\idFun \comp g
\equiv g$ and thus close the goal.
%
\subsection{Equality of isomorphic types}
%
Let $\top$ denote the unit type -- a type with a single constructor. In the
propositions-as-types interpretation of type theory $\top$ is the proposition
that is always true. The type $A \x \top$ and $A$ has an element for each $a :
A$. So in a sense they have the same shape (Greek; \nomen{isomorphic}). The
second element of the pair does not add any ``interesting information''. It can
be useful to identify such types. In fact, it is quite commonplace in
mathematics. Say we look at a set $\{x \mid \phi\ x \land \psi\ x\}$ and somehow
conclude that $\psi\ x \equiv \top$ for all $x$. A mathematician would
immediately conclude $\{x \mid \phi\ x \land \psi\ x\} \equiv \{x \mid
\phi\ x\}$ without thinking twice. Unfortunately such an identification can not
be performed in ITT.

More specifically what we are interested in is a way of identifying
\nomen{equivalent} types. I will return to the definition of equivalence later
in section \S\ref{sec:equiv}, but for now it is sufficient to think of an
equivalence as a one-to-one correspondence. We write $A \simeq B$ to assert that
$A$ and $B$ are equivalent types. The principle of univalence says that:
%
$$\mathit{univalence} \tp (A \simeq B) \simeq (A \equiv B)$$
%
In particular this allows us to construct an equality from an equivalence
($\mathit{ua} \tp (A \simeq B) \to (A \equiv B)$) and vice-versa.

\section{Formalizing Category Theory}
%
The above examples serve to illustrate a limitation of ITT. One case where these
limitations are particularly prohibitive is in the study of Category Theory. At
a glance category theory can be described as ``the mathematical study of
(abstract) algebras of functions'' (\cite{awodey-2006}). By that token
functional extensionality is particularly useful for formulating Category
Theory. In Category theory it is also common to identify isomorphic structures
and univalence gives us a way to make this notion precise. In fact we can
formulate this requirement within our formulation of categories by requiring the
\emph{categories} themselves to be univalent as we shall see.

\section{Context}
\label{sec:context}
%
The idea of formalizing Category Theory in proof assistants is not new. There
are a multitude of these available online. Just as a first reference see this
question on Math Overflow: \cite{mo-formalizations}. Notably these
implementations of category theory in Agda:
%
\begin{itemize}
\item
  \url{https://github.com/copumpkin/categories}

  A formalization in Agda using the setoid approach
\item
  \url{https://github.com/pcapriotti/agda-categories}

  A formalization in Agda with univalence and functional extensionality as
  postulates.
\item
  \url{https://github.com/HoTT/HoTT/tree/master/theories/Categories}

  A formalization in Coq in the homotopic setting
\item
  \url{https://github.com/mortberg/cubicaltt}

  A formalization in CubicalTT - a language designed for cubical-type-theory.
  Formalizes many different things, but only a few concepts from category
  theory.

\end{itemize}
%
The contribution of this thesis is to explore how working in a cubical setting
will make it possible to prove more things and to reuse proofs and to try and
compare some aspects of this formalization with the existing ones.\TODO{How can
  I live up to this?}

There are alternative approaches to working in a cubical setting where one can
still have univalence and functional extensionality. One option is to postulate
these as axioms. This approach, however, has other shortcomings, e.g.; you lose
\nomen{canonicity} (\TODO{Pageno!} \cite{huber-2016}). Canonicity means that any
well-typed term evaluates to a \emph{canonical} form. For example for a closed
term $e \tp \bN$ it will be the case that $e$ reduces to $n$ applications of
$\mathit{suc}$ to $0$ for some $n$; $e = \mathit{suc}^n\ 0$. Without canonicity
terms in the language can get ``stuck'' -- meaning that they do not reduce to a
canonical form.

Another approach is to use the \emph{setoid interpretation} of type theory
(\cite{hofmann-1995,huber-2016}). With this approach one works with
\nomen{extensional sets} $(X, \sim)$, that is a type $X \tp \MCU$ and an
equivalence relation $\mathord{\sim} \tp X \to X \to \MCU$ on that type. Under the setoid
interpretation the equivalence relation serve as a sort of ``local''
propositional equality. This approach has other drawbacks; it does not satisfy
all propositional equalities of type theory (\TODO{Citation needed}), is
cumbersome to work with in practice (\cite[p. 4]{huber-2016}) and makes
equational proofs less reusable since equational proofs $a \sim_{X} b$ are
inherently `local' to the extensional set $(X , \sim)$.

\section{Conventions}
\TODO{Talk a bit about terminology. Find a good place to stuff this little
  section.}

In the remainder of this paper I will use the term \nomen{Type} to describe --
well, types. Thereby diverging from the notation in Agda where the keyword
\texttt{Set} refers to types. \nomen{Set} on the other hand shall refer to the
homotopical notion of a set. I will also leave all universe levels implicit.

And I use the term \nomen{arrow} to refer to morphisms in a category, whereas
the terms morphism, map or function shall be reserved for talking about
type-theoretic functions; i.e. functions in Agda.

$\defeq$ will be used for introducing definitions. $=$ will be used to for
judgmental equality and $\equiv$ will be used for propositional equality.

All this is summarized in the following table:

\begin{center}
\begin{tabular}{ c c c }
Name & Agda & Notation \\
\hline
\nomen{Type}            & \texttt{Set}         & $\Type$            \\
\nomen{Set}             & \texttt{Σ Set IsSet} & $\Set$             \\
Function, morphism, map & \texttt{A → B}       & $A → B$            \\
Dependent- ditto        & \texttt{(a : A) → B} & $∏_{a \tp A} B$  \\
\nomen{Arrow}           & \texttt{Arrow A B}   & $\Arrow\ A\ B$     \\
\nomen{Object}          & \texttt{C.Object}    & $̱ℂ.Object$     \\
Definition              & \texttt{=}           & $̱\defeq$       \\
Judgmental equality     & \null                & $̱=$            \\
Propositional equality  & \null                & $̱\equiv$
\end{tabular}
\end{center}
