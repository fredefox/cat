\documentclass{article}



\usepackage[utf8]{inputenc}

\usepackage{natbib}
\usepackage[hidelinks]{hyperref}

\usepackage{graphicx}
\usepackage{color,soul}

\usepackage[colorinlistoftodos]{todonotes}

\usepackage{parskip}
\usepackage{multicol}
\usepackage{amsmath,amssymb}
% \setlength{\parskip}{10pt}

% \usepackage{tikz}
% \usetikzlibrary{arrows, decorations.markings}

% \usepackage{chngcntr}
% \counterwithout{figure}{section}

\usepackage{chalmerstitle}
\newcommand{\coloneqq}{\mathrel{\vcenter{\baselineskip0.5ex \lineskiplimit0pt
                     \hbox{\scriptsize.}\hbox{\scriptsize.}}}%
                     =}

\newcommand{\defeq}{\coloneqq}
\newcommand{\bN}{\mathbb{N}}
\newcommand{\bC}{\mathbb{C}}
\newcommand{\bX}{\mathbb{X}}
% \newcommand{\to}{\rightarrow}
\newcommand{\mto}{\mapsto}
\newcommand{\UU}{\ensuremath{\mathcal{U}}\xspace}
\let\type\UU
\newcommand{\nomen}[1]{\emph{#1}}
\newcommand{\todo}[1]{\textit{#1}}
\newcommand{\comp}{\circ}
\newcommand{\x}{\times}
\newcommand{\Hom}{\mathit{Hom}}
\newcommand{\fmap}{\mathit{fmap}}
\newcommand{\idFun}{\mathit{id}}
\newcommand{\Sets}{\mathit{Sets}}
\newcommand{\Set}{\mathit{Set}}
\newcommand{\MCU}{\UU}
\newcommand{\isSet}{\mathit{isSet}}
\newcommand{\tp}{\;\mathord{:}\;}
\newcommand{\subsubsubsection}[1]{\textbf{#1}}
\newcommand{\WIP}[1]{\textbf{[WIP]}}

% \newcommand{\sectiondescription}[1]{\todo[inline,color=green!40]{#1}}
\newcommand{\sectiondescription}[1]{\iffalse #1\fi}
\newcommand{\mycomment}[1]{\hl{#1}}

\title{Category Theory and Cubical Type Theory}
\author{Frederik Hanghøj Iversen}
\authoremail{hanghj@student.chalmers.se}
\supervisor{Thierry Coquand}
\supervisoremail{coquand@chalmers.se}
\cosupervisor{Andrea Vezzosi}
\cosupervisoremail{vezzosi@chalmers.se}
\institution{Chalmers University of Technology}

\begin{document}

\maketitle
%
\mycomment{Text marked like this are todo-comments.}
\sectiondescription{Text marked like this describe what should go in the section.}
%
\section{Introduction}
%
\sectiondescription{%
Briefly describe and motivate the project, and convince the reader of the
importance of the proposed thesis work. A good introduction will answer these
questions: Why is addressing these challenges significant for gaining new
knowledge in the studied domain? How and where can this new knowledge be
applied?
}
%
Functional extensionality and univalence is not expressible in
\nomen{Intensional Martin Löf Type Theory} (ITT). This poses a severe limitation
on both 1) what is \emph{provable} and 2) the \emph{reusability} of proofs.
Recent developments have, however, resulted in \nomen{Cubical Type Theory} (CTT)
which permits a constructive proof of these two important notions.

Furthermore an extension has been implemented for the proof assistant Agda that
allows us to work in such a ``cubical setting''. This project will be concerned
with exploring the usefulness of this extension. As a case-study I will consider
\nomen{category theory}. This case-study will serve a dual purpose: First off
category theory is a field where the notion of functional extensionality and
univalence wil be particularly useful. Secondly, Category Theory gives rise to
a \nomen{model} for CTT.

The project will consist of two parts: The first part will be concerned with
formalizing concepts from category theory. The focus will be on formalizing
parts that will be useful in the second part of the project: Showing that
\nomen{Cubical Sets} give rise to a \emph{model} for CTT.
%
\section{Problem}
%
\sectiondescription{%
This section is optional. It may be used if there is a need to describe the
problem that you want to solve in more technical detail and if this problem
description is too extensive to fit in the introduction.
}
%
In the following two subsections I present two examples that illustrate the
limitaiton inherent in ITT and by extension to the expressiveness of Agda.
%
\subsection{Functional extensionality}
Consider the functions:
%
\begin{multicols}{2}
$f \defeq (n : \bN) \mapsto (0 + n : \bN)$

$g \defeq (n : \bN) \mapsto (n + 0 : \bN)$
\end{multicols}
%
$n + 0$ is definitionally equal to $n$. We call this \nomen{defnitional equality}
and write $n + 0 = n$ to assert this fact. We call it definitional
equality because the \emph{equality} arises from the \emph{definition} of $+$
which is:
%
\newcommand{\suc}[1]{\mathit{suc}\ #1}
\begin{align*}
  +           & : \bN \to \bN              \\
  n + 0       & \defeq n                   \\
  n + (\suc{m}) & \defeq \suc{(n + m)}
\end{align*}
%
Note that $0 + n$ is \emph{not} definitionally equal to $n$. $0 + n$ is in
normal form. I.e.; there is no rule for $+$ whose left-hand-side matches this
expression. We \emph{do}, however, have that they are propositionally equal. We
write $n + 0 \equiv n$ to assert this fact. Propositional equality means that
there is a proof that exhibits this relation. Since equality is a transitive
relation we have that $n + 0 \equiv 0 + n$.

Unfortunately we don't have $f \equiv g$.\footnote{Actually showing this is
outside the scope of this text. Essentially it would involve giving a model
for our type theory that validates all our axioms but where $f \equiv g$ is
not true.} There is no way to construct a proof asserting the obvious
equivalence of $f$ and $g$ -- even though we can prove them equal for all
points. This is exactly the notion of equality of functions that we are
interested in; that they are equal for all inputs. We call this
\nomen{pointwise equality}. Where the \emph{points} of a function refers
to it's arguments.
%
\iffalse
I also want to talk about:
\begin{itemize}
\item
  Foundational systems
\item
  Theory vs. metatheory
\item
  Internal type theory
\end{itemize}
\fi
\subsection{Equality of isomorphic types}
%
The type $A \x \top$ and $A$ has an element for each $a : A$. So in a sense they
are the same. The second element of the pair does not add any ``interesting
information''. It can be useful to identify such types. In fact, it is quite
commonplace in mathematics. Say we look at a set $\{x \mid \phi\ x \land
\psi\ x\}$ and somehow conclude that $\psi\ x \equiv \top$ for all $x$. A
mathematician would immediately conclude $\{x \mid \phi\ x \land \psi\ x\}
\equiv \{x \mid \phi\ x\}$ without thinking twice. Unfortunately such an
identification can not be performed in ITT. 

More specifically; what we are interested in is a way of identifying types that
are in a one-to-one correspondence. We say that such types are
\nomen{isomorphic} and write $A \cong B$ to assert this.

To prove an isomorphism is give an \nomen{isomorphism} between the two types.
That is, a function $f : A \to B$ for which it has an inverse $f^{-1} : B \to
A$, i.e.: $f^{-1} \comp f \equiv id_A$. If such a function exist we say that $A$
and $B$ are isomorphic and write $A \cong B$.

What we want is to identify isomorphic types. This is the principle of
univalence:\footnote{It's often referred to as the univalence axiom, but since
it is not an axiom in this setting but rather a theorem I refer to this just
as a `principle'.}
%
$$(A \cong B) \cong (A \equiv B)$$
%
\subsection{Category Theory as a case-study}
%
The above examples serves to illustrate the limitation of Agda. One case where
these limitations are particularly prohibitive is in the case of Category
Theory. Category Theory -- at a glance -- is ``is the mathematical study of
(abstract) algebras of functions'' (\cite{awodey-2006}). So by that token
functional extensionality is particularly useful for formulating Category
Theory. Another aspect of Category Theory is that one usually want to talk about
things ``up to isomorphism''. Another way of phrasing this is that we want to
identify isomorphic objects. This is exactly what we get from univalence.

\mycomment{Can there be issues with identifying isomoprhic types? Suddenly many
seemingly different objects collaps into the same thing (e.g.: $\{1\} \equiv
\{2\}$, $\mathbb{Z} \equiv \mathbb{N}$, \ldots)}

\subsection{Category Theory as a model for Cubical Type Theory}
%
Certain categories give rise to a model for Cubical Type Theory
(\cite{bezem-2014}). \cite{dybjer-1995} describe how to construct a model for a
type theory. One part of which is to `check equations' - that is, that the model
satisfies the axioms -- i.e. typing rules -- for the type theory under study. In
this case the Cubical Sets have to satisfy the corresponding `translation' of
those axioms in the categorical setting. The \emph{translation} will not be
given formally (i.e. as a function in Agda). That translation will be given
informally, and I will show that the model satisfies these.
%
\mycomment{Quickly explain that we can formulate the language of Cubical Type
Theory and show that Cubical Sets are a model of this.}
%
\section{Context}
%
\sectiondescription{%
Use one or two relevant and high quality references for providing evidence from
the literature that the proposed study indeed includes scientific and
engineering challenges, or is related to existing ones. Convince the reader that
the problem addressed in this thesis has not been solved prior to this project.
}
%
Work by \citeauthor{bezem-2014} resulted in a model for type theory where
univalence is expressible. This model is an example of a \nomen{categorical
model} -- that is, a model formulated in terms of categories. As such this
paper will also serve as a further object of study for the concepts from
Category Theory that I will have formalized.

Work by \citeauthor{cohen-2016} have resulted in a type system where univalence
is expressible. The categorical model from above is a model of this type theory.
So these two ideas are closely related.

An implementation of cubical type theory can be found as an extension to Agda.
This is due to \citeauthor{cubical-agda}. This, of course, will be central to
this thesis. As such, my work with this extension will serve as evidence to the
merrit of this implementation.

The idea of formalizing Category Theory in proof assistants is not a new idea
(\mycomment{citations \ldots}). The contribution of this thesis is to explore
how working in a cubical setting will make it possible to proove more things and
to resuse proofs. There are alternative approaches to working in a cubical
setting where one can still have univalence and functional extensionality. One
could e.g. postulate these as axioms. This approach has other shortcomings,
e.g.; you loose canonicity (\mycomment{citation}). \mycomment{Perhaps we could
  also formulate equality as another type. What are some downsides of this
  approach?}

\mycomment{Mention internal type theory c.f. Dybjers paper? He talks about two
types of internal type theory. One of them is where you express the typing
rules of your languages within that languages}

\mycomment{Other aspects that I think are interesting: Type Theory as a foundational
system; why is ``nice'' to have a categorical model?}

\mycomment{Should perhaps mention how Cubical Type Theory came out out of Homotopy
Type Theory that came out of Topology}
%
\section{Goals and Challenges}
%
\sectiondescription{%
Describe your contribution with respect to concepts, theory and technical goals.
Ensure that the scientific and engineering challenges stand out so that the
reader can easily recognize that you are planning to solve an advanced problem.
}
%
In summary, the aim of the project is to:
%
\begin{itemize}
\item
Formalize Category Theory in Cubical Agda
\item
Formalize Cubical Sets in Agda
% \item
% Formalize Cubical Type Theory in Agda
\item
Show that Cubical Sets are a model for Cubical Type Theory
\end{itemize}
%
The formalization of category theory will focus on extracting the elements from
Category Theory that we need in the latter part of the project. In doing so I'll
be gaining experience with working with Cubical Agda. Equality proofs using
cubical Agda can be tricky, so working with that will be a challenge in itself.
Most of the proofs I will do will be based on previous work. These proofs are
pen-and-paper proof. Translating such proofs to a type system is not always
straight-forward. A further challenge is that in cubical Agda there can be
multiple distinct terms that inhabit a given equality proof. This means that the
choice for a given equality proof can influence later proofs that refer back to
said proof. This is new and relatively unexplored territory. Another challenge
is that Category Theory is something that I only know the basics of. So learning
the necessary concepts from Category Theory will also be a goal and a challenge
in itself.

After this has been implemented it would also be possible to formalize Cubical
Type Theory and formally show that Cubical Sets are a model of this. This is not
a strict goal for this thesis but would certainly be a natural extension of it.

The final thesis should also include a discussion about the pros/cons of using
Cubical Agda; \mycomment{have Agda become more useful, easy to work with,
\ldots ? } Ideally my work will serve as an argument that working in a Cubical
setting is useful.
%
\section{Approach}
%
\sectiondescription{%
Various scientific approaches are appropriate for different challenges and
project goals. Outline and justify the ones that you have selected. For example,
when your project considers systematic data collection, you need to explain how
you will analyze the data, in order to address your challenges and project
goals.

One scientific approach is to use formal models and rigorous mathematical
argumentation to address aspects like correctness and efficiency. If this is
relevant, describe the related algorithmic subjects, and how you plan to address
the studied problem. For example, if your plan is to study the problem from a
computability aspect, address the relevant issues, such as algorithm and data
structure design, complexity analysis, etc. If you plan to develop and evaluate
a prototype, briefly describe your plans to design, implement, and evaluate your
prototype by reviewing at most two relevant issues, such as key functionalities
and their evaluation criteria.

The design and implementation should specify prototype properties, such as
functionalities and performance goals, e.g., scalability, memory, energy.
Motivate key design selection, with respect to state of the art and existing
platforms, libraries, etc.

When discussing evaluation criteria, describe the testing environment, e.g.,
test-bed experiments, simulation, and user studies, which you plan to use when
assessing your prototype. Specify key tools, and preliminary test-case
scenarios. Explain how and why you plan to use the evaluation criteria in order
to demonstrate the functionalities and design goals. Explain how you plan to
compare your prototype to the state of the art using the proposed test-case
evaluation scenarios and benchmarks.
}
%
\mycomment{I don't know what more I can say here than has already been
explained. Perhaps this section is not needed for me?}
%
\section{References}
%
\bibliographystyle{plainnat}

\bibliography{refs} \mycomment{I have a bunch of other relevant references that
I haven't been able to incorporate into my text yet...}
\end{document}
